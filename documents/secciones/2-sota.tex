% ================================================
% =              STATE OF THE ART                =
% ================================================ 

La conducción automatizada es un campo que está ganando cada vez más atención y que está en constate crecimiento. Existen varios artículos de revisión del estado del arte \cite{survey_AutomatedDriving1} \cite{survey_AutomatedDriving2}... en los que se muestra cómo la conducción automatizada lleva presente más de 20 años, desde los DARPA Challenges en 2003 y 2005 o el Grand DARPA Urban challenge de 2007 en los que los vehículos ... hasta las últimos avances gracias al progreso en el área del DeepLearning y el Machine Learning.

Además, el uso de \aclink{ADSs} parece indicar en varios reports un descenso en el índice de accidentes causados por erroes humanos \hl{Esto porque Miami me lo confirmó... quieres decir muchas cosas asique organiza}.

En el desarrollo de los \aclink{ADSs} intervienen numerosas disciplinas, desde robótica y automoción hasta la disponibilidad de nuevos sensores y los últimos avances en Computer Vision impulsados principalmente por el aprendizaje automático y el aprendizaje profundo.

En este contexto, la \aclink{SAE} ha definido 5 niveles de automatizacíon en los vehículos, aportando un marco de desarrollo... \hl{Explicar por encima los niveles SAE}

Existen varias arquitecturas para los \aclink{ADSs}. En \cite{survey_AutomatedDriving1} se presenta una clasificación basada en la conectividad y en el diseño de los algoritmos. Existen desde soluciones End-to-End, que combinan técnicas de aprendizaje profundo y aprendizaje por refuerzo para obtener las comandas de control del vehículo partiendo diréctamente de los datos de los sensores; hasta soluciones modulares más tradicionales que dividen el problema de la conducción automatizada en sub-tasks en los que integran disciplinas de otros campos como la robótica, computer vision o control automático...  \hl{Explicar connected vehicles}

Se han desarrollado numerosas arquitecturas para sistemas \aclink{ADSs} modulares. Sin embargo, estos sub-tasks se pueden categorizar en tres grupos principales \cite{machines5010006}: percepción, planificación y control con interacciones entre ellos, el hardware del vehículo y otras comunicaciones como V2I o V2X (connected vehicles)... \hl{Explicar cada uno de ellos}



