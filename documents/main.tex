\documentclass[12pt, a4paper]{article}
\usepackage[spanish]{babel}             % Español para los nombres
\usepackage[utf8]{inputenc}             % Codificación de entrada
\usepackage[T1]{fontenc}                % Codificación de fuente (escribir castellano)
\usepackage{lmodern}                    % Fuente (la default no es compatible con castellano)
\usepackage{csquotes}                   % Para comillas y otras anotaciones del castellano
\usepackage{fancyhdr}                   % Encabezado y Pie de pagina
\usepackage{tocloft}                    % Puntos en los índices

\usepackage{url}                        % 
\usepackage{hyperref}                   % Links y referencias dentro del texto
\usepackage{nameref}                    % Para poder referenciar por nombre
\usepackage{acronym}                    % Para incluir las listas de abreviaturas
\usepackage{multicol}                   % Permite crear espacios con columnas verticales
\usepackage{booktabs} % Para mejores líneas horizontales en tablas
\usepackage{parskip}                    % Eliminar sangrado y añadir espacio entre párrafos
\usepackage{titlesec}                   % Crear subsubsub secciones
\usepackage{amsmath}                    % Funciones para ecuaciones
\usepackage{blkarray}                   % Matrices con anotaciones
\usepackage{algorithm}                  % Algoritmos con formalismos
\usepackage{algpseudocode}              % También para algoritmos
\usepackage{listings}                   % Para insertar codigo

\usepackage{array}                      % Para personalizar columnas de tabla
\usepackage{tabularx}                   % Para tablas de ancho variable

\usepackage{graphicx}                   % Para incluir imágenes
\usepackage{subcaption}                 % Subfiguras
\usepackage[table]{xcolor}              % Definir y utilizar colores
\usepackage{tikz}                       % Dibujar formas, figuras, rectas, intersecciones...
\usepackage[absolute,overlay]{textpos}  % Posición absoluta para textos
\usepackage{geometry}                   % Para controlar los márgenes

\usepackage{float}


\usepackage[backend=biber, sorting=none]{biblatex} % Referencias (bibliografía / webgrafía)
\selectlanguage{spanish}                % Seleccionar español
\bibliography{referencias}              % Incluir el archivo de referencias

% Definición de nombres
%\printbibliography[title={Bibliografía}]
\renewcommand{\cftsecleader}{\cftdotfill{\cftdotsep}} 
\renewcommand{\spanishtablename}{Tabla}
\renewcommand{\spanishlisttablename}{Índice de tablas}
\newcommand{\estudiante}{García Justel, Alan}
\newcommand{\director}{}
\newcommand{\master}{INGENIERÍA COMPUTACIONAL Y SISTEMAS INTELIGENTES}
\newcommand{\titulo}{PREANOTACIÓN DE ESCENAS VEHICULARES CON SEGMENTACIÓN SEMÁNTICA EN VISTA TOP-DOWN}
\newcommand{\portada}{images/shared/no_signal.jpg}
\newcommand{\curso}{2023-2024}

% Definición de Macros
\newcommand*{\fullref}[1]{\hyperref[{#1}]{\autoref*{#1} \nameref*{#1}}} % One single link

\newcommand{\aclink}[1]{% Crear links a acronimos
  \ifcsdef{ac@#1}{%
    \ifcsdef{mark@#1}{%
      \hyperlink{acro:#1}{\acs{#1}}%
    }{%
      \hyperlink{acro:#1}{\ac{#1}}%
      \expandafter\gdef\csname mark@#1\endcsname{}%
    }%
  }{%
    \ac{#1}%
    \expandafter\gdef\csname mark@#1\endcsname{}%
  }%
}

% Notas en Figuras
\newcommand\fnote[1]{\captionsetup{width=0.8\linewidth, font=footnotesize}\caption*{#1}}

% Estilos de listings
% Configuración del estilo para resaltar el código de ROS
\lstdefinestyle{ros}{
    backgroundcolor=\color{white}, % Cambiar el color de fondo
    basicstyle=\small\ttfamily, % Texto más pequeño
    breaklines=true,
    language=C,
    morekeywords={int32, float64},
    keywordstyle=\color{blue},
    commentstyle=\color{green!40!black},
    stringstyle=\color{red},
    breakatwhitespace=false,         
    breaklines=true,                 
    captionpos=b,                    
    keepspaces=true,                 
    numbers=left,                    
    numbersep=5pt,                  
    showspaces=false,                
    showstringspaces=false,
    showtabs=false,                  
    tabsize=2
}

% Crear subsubsub secciones
\titleclass{\subsubsubsection}{straight}[\subsection]
\newcounter{subsubsubsection}[subsubsection]
\renewcommand\thesubsubsubsection{\thesubsubsection.\arabic{subsubsubsection}}
\renewcommand\theparagraph{\thesubsubsubsection.\arabic{paragraph}} % optional; useful if paragraphs are to be numbered

\titleformat{\subsubsubsection}
  {\normalfont\normalsize\bfseries}{\thesubsubsubsection}{1em}{}
\titlespacing*{\subsubsubsection}
{0pt}{3.25ex plus 1ex minus .2ex}{1.5ex plus .2ex}
\makeatletter
\renewcommand\paragraph{\@startsection{paragraph}{5}{\z@}%
  {3.25ex \@plus1ex \@minus.2ex}%
  {-1em}%
  {\normalfont\normalsize\bfseries}}
\renewcommand\subparagraph{\@startsection{subparagraph}{6}{\parindent}%
  {3.25ex \@plus1ex \@minus .2ex}%
  {-1em}%
  {\normalfont\normalsize\bfseries}}
\def\toclevel@subsubsubsection{4}
\def\toclevel@paragraph{5}
\def\toclevel@paragraph{6}
\def\l@subsubsubsection{\@dottedtocline{4}{7em}{4em}}
\def\l@paragraph{\@dottedtocline{5}{10em}{5em}}
\def\l@subparagraph{\@dottedtocline{6}{14em}{6em}}
\makeatother
\setcounter{secnumdepth}{4}
\setcounter{tocdepth}{4}



% Encabezado y Pie de pagina
\pagestyle{fancy}                       % Estilo de las páginas
\fancyhf{}
\fancyhead[L]{Trabajo Fin de Máster}
\fancyhead[R]{\leftmark}
\fancyfoot[L]{UPV/EHU}
\fancyfoot[R]{\thepage}
\renewcommand{\footrulewidth}{0.4pt} % Línea horizontal en el pie de página
\setlength{\headheight}{15.5pt}

% Definición de colores
\definecolor{ehu_blue}{HTML}{376092}
\definecolor{link_color}{HTML}{36AEB4}
\definecolor{reference_color}{HTML}{0F3133}
\definecolor{cite_color}{HTML}{ff7f42}
\definecolor{table_gray}{HTML}{C0C0C0}
\definecolor{table_red}{HTML}{fb4952}


% Incluye todas las entradas del archivo .bib en la bibliografía
\nocite{*}

% Configuración de hyperref para diferentes tipos de enlaces
\hypersetup{
    colorlinks=true,
    linkcolor=reference_color,
    citecolor=cite_color,
    filecolor=link_color,
    urlcolor=link_color,
}

% #########################################################################
% #                                  TFM                                  #
% #########################################################################
\begin{document}
\newgeometry{bottom=2cm}

\begin{titlepage}
    % Logo de la universidad
    \begin{textblock*}{\textwidth}(10cm,0cm)
        \includegraphics[width=7.5cm, height=3cm]{images/shared/Logo_EHU.jpg}
    \end{textblock*}
    
    % Franja azul
    \begin{tikzpicture}[remember picture, overlay]
        \fill[ehu_blue] (current page.north west) ++ (0,-3.01cm) rectangle (\paperwidth,-3cm);
    \end{tikzpicture}
    
    \begin{textblock*}{\paperwidth}(\dimexpr\parindent+\oddsidemargin+3em\relax,3.5cm)
        \begin{minipage}{\dimexpr\linewidth-7.5cm\relax}
            \color{white}
            \noindent\rule{\linewidth}{0cm}
            \textsf{ {\large MÁSTER EN \master}}
            \newline
            \newline \newline
            \textsf{\textbf{ {\Huge TRABAJO FIN DE MÁSTER }}}
        \end{minipage}
    \end{textblock*}
    
    % Título del trabajo
    \vspace*{3.5cm}
    \begin{minipage}{\linewidth}
        \setlength{\baselineskip}{1.7\baselineskip}
        \centering
        \textsf{ \textbf{ {\LARGE \titulo }}}
    \end{minipage}

    % Foto de portada
    \vspace*{0.5cm}
    \begin{figure}[H]
        \centering
        \includegraphics[width=10cm, height=8cm]{\portada}
    \end{figure}

    % ODS
    \vspace*{1cm}
    % \begin{figure}[h]
    % \centering
    %     \begin{subfigure}[b]{0.135\textwidth}
    %         \includegraphics[width=2cm, height=2cm]{images/iconos_ods/03.png}
    %     \end{subfigure}
    %     \begin{subfigure}[b]{0.135\textwidth}
    %         \includegraphics[width=2cm, height=2cm]{images/iconos_ods/04.png}
    %     \end{subfigure}
    %     \begin{subfigure}[b]{0.135\textwidth}
    %         \includegraphics[width=2cm, height=2cm]{images/iconos_ods/05.png}
    %     \end{subfigure}
    %     \begin{subfigure}[b]{0.135\textwidth}
    %         \includegraphics[width=2cm, height=2cm]{images/iconos_ods/07.png}
    %     \end{subfigure}
    %     \begin{subfigure}[b]{0.135\textwidth}
    %         \includegraphics[width=2cm, height=2cm]{images/iconos_ods/08.png}
    %     \end{subfigure}
    %     \begin{subfigure}[b]{0.135\textwidth}
    %         \includegraphics[width=2cm, height=2cm]{images/iconos_ods/09.png}
    %     \end{subfigure}
    %     \begin{subfigure}[b]{0.135\textwidth}
    %         \includegraphics[width=2cm, height=2cm]{images/iconos_ods/12.png}
    %     \end{subfigure}
    % 
    %     \label{fig:ods-iconos}
    % \end{figure}
    
    % Estudiante
    \vspace{0.2cm}
    \noindent {\footnotesize \textbf{Estudiante:} \estudiante}
    \newline
    \noindent\makebox[\linewidth]{\rule{\textwidth}{0.4pt}} % Línea horizontal

    % Director
    \nopagebreak
    \vspace{0.3cm}
    \nopagebreak
    \noindent {\footnotesize \textbf{Director/Directora:} \director }

    % Espacio para firmas
    \vspace{0.5cm} % Espacio entre texto "Director/Directora" y espacio para firmas
    \noindent 
    \makebox[0.4\linewidth]{\hrulefill}
    \hspace{0.2\linewidth}
    \makebox[0.4\linewidth]{\hrulefill}

    % Curso y Fecha
    \vspace{0.1cm}    
    \noindent {\footnotesize \textbf{Curso: } \curso \hfill \textbf{Fecha:} \today }
\end{titlepage}

\restoregeometry
\setcounter{figure}{0} % Incluimos el título

% \newpage % BORRAR EN LA VERSIÓN FINAL
% \input{notas}

\newpage
\markboth{Resumen}{Resumen}
\begin{itshape}
    \textbf{Resumen:} \\
    
    \textbf{Palabras Clave: }
\end{itshape}
\newpage

\begin{itshape}
    \textbf{Abstract:} \\

    \textbf{Key Words: }
\end{itshape}
\newpage

\begin{itshape}
    \textbf{Laburpena:} \\

    \textbf{Gako-hitzak: }
\end{itshape}

\newpage

 % Incluimos el resumen tri-lingüe

% Índices
\tableofcontents\thispagestyle{empty}\newpage
\listoffigures\thispagestyle{empty}\newpage
\listoftables\thispagestyle{empty}\newpage

% Abreviaturas
\section*{Abreviaturas}
\markboth{Abreviaturas}{Abreviaturas}
\addcontentsline{toc}{section}{Abreviaturas}
%\begin{multicols}{2} % Si se quieren en dos columnas verticales
\begin{acronym}[ASSI]  % El acrónimo más largo
    \acro{TFM}{Trabajo Fin de Máster}
    \acro{ADSs}{Automated Driving Systems}
    \acro{SAE}{Society of Automotive Engineers}
\end{acronym}
%\end{multicols}

 % Marcar como visto (No te desglosa el acrónimo)
 % Provoca warnings pero funciona :)
% \acused{ADAS}

\newpage

\section{Introducción}
% ================================================
% =                INTRODUCCIÓN                  =
% ================================================ 
bla bla bla

% \newpage
% \section*{Agradecimientos}
% % ==========================================================
% =                    AGRADECIMIENTOS                     =
% ==========================================================
\label{agradecimiento}
Lore ipsum...
% 
% \newpage
% \section{Anexos}
% % ==========================================================
% =                         ANEXOS                         =
% ==========================================================
\phantomsection
\subsection{Dummy section} 
Lore ipsum...



\newpage\printbibliography
\addcontentsline{toc}{section}{Referencias}

\end{document}
